\documentclass[a4paper, 12pt]{article}
\usepackage[T2A]{fontenc}
\usepackage[utf8]{inputenc}
\usepackage[russian]{babel}
\usepackage[left=2cm,right=2cm,top=2cm,bottom=2cm]{geometry} 

\usepackage{amsmath}
\usepackage{amsfonts}
\usepackage{amssymb}
\usepackage{graphicx}
\usepackage{tabularx}
\usepackage{indentfirst}

%Колонтитулы
\usepackage{lastpage} 
\usepackage{fancybox,fancyhdr}
\fancyhead[R]{Справочник формул по математике}
\fancyhead[L]{Владимир Анатольевич Калитвин}
\fancyhead[C]{}
\fancyfoot[R]{Калитвин В.А. (kalitvin@gmail.com)}
\fancyfoot[L]{Страница \thepage \; из \pageref{LastPage}}
\fancyfoot[C]{}
\pagestyle{fancy}
%

%Рамки
\usepackage{framed}
%

\begin{document}
\author{Калитвин В.А.\\
kalitvin@gmail.com}
\title{Справочник формул по математике}
\maketitle
\thispagestyle{empty}
\newpage
\section{Формулы сокращенного умножения}

Квадрат суммы
$$ (a+b)^2=a^2+2ab+b^2$$

$$ (a+b+c)^2=a^2+b^2+c^2+2ab+2bc+2ac$$

Квадрат разности
$$(a-b)^2=a^2-2ab+b^2$$

Куб суммы

$$(a+b)^3=a^3+3a^2b+3ab^2+b^3$$

Куб разности
$$(a-b)^3=a^3-3a^2b+3ab^2-b^3$$

Разность квадратов
$$a^2-b^2=(a-b)(a+b)$$

Сумма кубов
$$a^3+b^3=(a+b)(a^2-ab+b^2)$$

Разность кубов
$$a^3-b^3=(a-b)(a^2+ab+b^2)$$

Для $n\in N$
$$a^n-b^n=(a-b)(a^{n-1}+a^{n-2}b+a^{n-3}b^2+\dots +ab^{n-2}+b^{n-1})$$

Если $n$ - четное
$$a^n-b^n=(a+b)(a^{n-1}-a^{n-2}b+a^{n-3}b^2-\dots +ab^{n-2}+b^{n-1})$$

Если $n$ - нечетное
$$a^n+b^n=(a+b)(a^{n-1}-a^{n-2}b+a^{n-3}b^2-\dots -ab^{n-2}+b^{n-1})$$

Бином Ньютона
$$(a+b)^n=\sum\limits_{k=0}^n C_n^ka^{n-k}b^k=$$
$$=C_n^0a^n+C_n^1a^{n-1}b^1+C_n^2a^{n-2}b^2+\dots +C_n^{n-1}a^1b^{n-1}+C_n^nb^n,$$
где $C_n^k=\frac{n!}{k!(n-k)!}$ --- число сочетаний из $n$ по $k.$

\section{Свойства степени}

$$
\begin{array}{ll}
a^0=1  & a^m\cdot a^n=a^{m+n}\\
a^m: a^n=a^{m-n} & a^{-n}=\frac{1}{a^n}\\
(a^m)^n=a^{mn} & (\frac{a}{b})^{-m}=(\frac{b}{a})^{m}\\
(a\cdot b)^m=a^m\cdot b^m & a^{\frac{1}{n}}=\root{n}\of{a}\\
(\frac{a}{b})^m =\frac{a^m}{b^m} & a^{\frac{m}{n}}=\root{n}\of{a^m}\\
\end{array}
$$

\section{Свойста квадратного (арифметического) корня}

$$
\begin{array}{ll}
\sqrt{a}\cdot \sqrt{b} = \sqrt{ab}  & \root{n}\of{a}=\root{nk}\of{a^k} \\
\frac{\sqrt{a}}{\sqrt{b}}=\sqrt{\frac{a}{b}}, b\not= 0 & \root{n}\of{a}\cdot \root{n}\of{b}=\root{n}\of{a\cdot b}\\
(\sqrt{a})^m=\sqrt{a^m} & \frac{\root{n}\of{a}}{\root{n}\of{b}}=\root{n}\of{\frac{a}{b}}, b\not=0\\
\sqrt{ab}=\sqrt{|a|}\cdot \sqrt{|b|} & (\root{n}\of{a})^m = \root{n}\of{a^m}\\
\sqrt{\frac{a}{b}}=\frac{\sqrt{|a|}}{|b|}, b\not=0 & \root{n}\of{\root{m}\of{a}} = \root{nm}\of{a}\\
\sqrt{a^m}=(\sqrt{|a|})^m
\end{array}
$$

\section{Тригонометрия}

\textbf{Тригонометрические тождества}

$$\sin^\alpha+\cos^2\alpha=1$$

$$\tg\alpha =\frac{\sin\alpha}{\cos\alpha}$$

$$\ctg\alpha =\frac{\cos\alpha}{\sin\alpha}$$

$$\tg\alpha \cdot \ctg\alpha = 1$$

$$|\cos \alpha |=\sqrt{1-\sin^2\alpha}$$

$$|\sin \alpha |=\sqrt{1-\cos^2\alpha}$$

$$\tg\alpha = \frac{1}{\ctg\alpha}$$

$$\ctg\alpha = \frac{1}{\tg\alpha}$$

$$1+\tg^2\alpha =\frac{1}{\cos^2\alpha}=sec^2\alpha$$

$$1+\ctg^2\alpha =\frac{1}{\sin^2\alpha}=cosec^2\alpha$$

\textbf{Формулы сложения тригонометрических функций}

$$\sin(\alpha \pm \beta )=\sin \alpha \cos \beta \pm \cos \alpha \sin \beta $$

$$\cos(\alpha \pm \beta )=\cos \alpha \cos\beta \mp \sin\alpha\sin\beta$$

$$\tg(\alpha\pm\beta)=\frac{\tg\alpha\pm\tg\beta}{1\mp\tg\alpha\tg\beta}$$

$$\ctg(\alpha\pm\beta)=\frac{\ctg\alpha\ctg\beta\mp1}{\ctg\beta\pm\ctg\alpha}$$

$$\tg x+\ctg y=\frac{\cos(x-y)}{\cos x\sin y}$$

$$\tg x -\ctg y=-\frac{cos(x+y)}{\cos x\sin y}$$

$$\tg x+\ctg x=\frac{1}{\sin x\cos x}=\frac{2}{\sin 2x}$$

$$\tg x-\ctg x=-2\frac{\cos 2x}{\sin 2x}=2\ctg 2x$$

$$\cos x + \sin x = \sqrt{2}\cos(45^\circ -x)=\sqrt{2}\sin(45^\circ +x)$$

$$\cos x - \sin x = \sqrt{2}\sin(45^\circ -x)=\sqrt{2}\cos(45^\circ +x)$$

$$a\sin x +b\cos x = \sqrt{a^2+b^2}\sin (x+\varphi), 
\hbox{ где } \sin\varphi = \frac{b}{\sqrt{a^2+b^2}}, \  \cos\varphi=\frac{a}{\sqrt{a^2+b^2}}$$

\textbf{Тригонометрические функции двойного аргумента}

$$\sin2\alpha=2\sin\alpha\cos\alpha$$
$$\cos2\alpha=\cos^2\alpha-\sin^2\alpha=1-2\sin^2\alpha=2\cos^2\alpha-1$$
$$\tg2\alpha=\frac{2\tg\alpha}{1-\tg^2\alpha}=\frac{2}{\ctg\alpha-\tg\alpha}$$
$$c\tg2\alpha=\frac{\ctg^2\alpha-1}{2\ctg\alpha}=\frac{\ctg\alpha-\tg\alpha}{2}$$

\textbf{Тригонометрические функции тройного аргумента}

$$\sin3\alpha=3\sin\alpha-4\sin^3\alpha$$
$$\cos3\alpha=4\cos^3\alpha - 3\cos\alpha$$
$$\tg3\alpha=\frac{3\tg\alpha-\tg^3\alpha}{1-3\tg^2\alpha}$$
$$\ctg3\alpha=\frac{\ctg^3\alpha-3\ctg\alpha}{3\ctg^2\alpha-1}$$

\textbf{Тригонометрические функции половиннго аргумента}

$$\sin^2\frac{\alpha}{2}=\frac{1-\cos\alpha}{2}$$
$$\cos^2\frac{\alpha}{2}=\frac{1+\cos\alpha}{2}$$
$$\tg^2\frac{\alpha}{2}=\frac{1-\cos\alpha}{1+\cos\alpha}$$
$$\ctg^2\frac{\alpha}{2}=\frac{1+\cos\alpha}{1-\cos\alpha}$$
$$\tg\frac{\alpha}{2}=\frac{\sin\alpha}{1+\cos\alpha}=\frac{1-\cos\alpha}{\sin\alpha}$$
$$\ctg\frac{\alpha}{2}=\frac{\sin\alpha}{1-\cos\alpha}=\frac{1+\cos\alpha}{\sin\alpha}$$

\textbf{Выражение тригонометрических функций через тангенс половинного угла}

$$\sin\alpha=\frac{2\tg\frac{\alpha}{2}}{1+\tg^2\frac{\alpha}{2}}$$
$$\cos\alpha=\frac{1-\tg^2\frac{\alpha}{2}}{1+\tg^2\frac{\alpha}{2}}$$
$$\tg\alpha=\frac{2\tg\frac{\alpha}{2}}{1-\tg^2\frac{\alpha}{2}}$$
$$\ctg\alpha=\frac{1-\tg^2\frac{\alpha}{2}}{2\tg\frac{\alpha}{2}}$$

\textbf{Формулы преобразования frпроизведения в сумму}
$$\sin x\sin y=\frac{1}{2}\left( \cos(x-y)-\cos(x+y)\right)$$
$$\cos x\cos y=\frac{1}{2}\left(\cos (x-y)+\cos (x+y)\right)$$
$$\sin x\cos y=\frac{1}{2}\left( \sin (x-y)+\sin (x+y)\right)$$

\section{Логарифмы}
\textbf{Определение логарифма.} Логарифмом положительного числа $b$ по основанию $a\ (a>0, a\not=1 )$ называется показатель степени, в которую нужно возвести $a$, чтобы получить $b.$

$$log_ab=c \Leftrightarrow a^c=b$$
 
\textbf{Свойства логарифма}

Основное логарифмическое тождество:
$$a^{log_ab}=b,  $$
$$\hbox{ где } a>0; a\not= 1; b>0.$$

$$log_aa=1$$

$$log_a1=0$$

$$log_aa^m=m$$

Логарифм произведения
$$log_c(ab)=log_ca+log_cb, \ x>0, y>0.$$

Логарифм частного
$$log_c(\frac{a}{b})=log_ca-log_cb, \ x>0, y>0$$

Логарифм степени
$$log_ca^n=nlog_ca, x>0.$$
$$log_{c^n}a=\frac{1}{n}log_ca, x>0.$$

Логарифм корня
$$log_c \root{n}\of{a}=\frac{1}{n}log_ca$$

Переход к новому основанию
$$log_ab=\frac{log_cb}{log_ca}, a>0, a\not=1, c>0, c\not=1, b>0$$

Формулы, следующие из свойств логарифмов
$$log_ab=\frac{1}{log_ba}$$
$$\frac{log_nb}{log_nc}=\frac{log_mb}{log_mc}=log_cb$$
$$log_nb\cdot log_mc=log_mb\cdot log_nc$$
$$a^{log_nb}=b^{log_na}$$

Десятичный логарифм - это логарифм по основанию 10:
$$log_{10}b=lgb$$

Натуральный логарифм --- это логарифм по основанию $e.$
$$log_eb=ln b.$$

\section{Текстовые задачи}
\subsection{Задачи на движение}
$$S=v\cdot t,$$
где $v$ --- скорость движения, $t$ --- время, $S$ --- расстояние, пройденное за время $t$ со скоростью $v.$
\subsection{Задачи на работу}
$$A=N\cdot t,$$
где $N$ --- работа, произведенная в единицу времени, $t$ --- время, в течение которого производится работа, $A$ --- работа, произведенная за время $t.$
\subsection{Задачи на сложные проценты}
$$A_n=A_0\left( 1+\frac{p}{100}\right)^n ,$$
где $A_0$ --- начальный капитал, $p\%$ --- процент годовыхм, $n$ --- годы, на которые положен вклад, $A_n$ --- наращенный капитал за $n$ лет. 
\subsection{Задачи на десятичную форму числа}
Стандартным видом числа $x$ называют его запись в виде $a\cdot 10^n,$ где $1\le a < 10$ и $n$ --целое число.

Число $n$ называют порядком числа $x.$ 

\subsection{Задачи на концентрацию смеси и сплавы}
\textbf{Процентными содержаниями} веществ $A, B, C$ в данной смеси называются величины $p_A\%, p_B\%, p_c\%,$ соответственно вычисляемые по формулам:
$$
p_A\%=C_A\cdot 100\%,\  p_B\%=C_B\cdot 100\%,\ p_C\%=C_C\cdot 100\%,
$$
где $C_A, C_B, C_C$ --- масса соответствующих веществ.

\end{document}