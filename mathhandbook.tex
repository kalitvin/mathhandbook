\documentclass[a4paper, 12pt]{article}
\usepackage[T2A]{fontenc}
\usepackage[utf8]{inputenc}
\usepackage[russian]{babel}
\begin{document}
\author{Калитвин В.А.}
\title{Справочник формул по математике}
\maketitle
\thispagestyle{empty}
\newpage
\section{Формулы сокращенного умножения}

Квадрат суммы
$$ (a+b)^2=a^2+2ab+b^2$$
Квадрат разности
$$(a-b)^2=a^2-2ab+b^2$$
Куб суммы
$$(a+b)^3=a^3+3a^2b+3ab^2+b^3$$
Куб разности
$$(a-b)^3=a^3-3a^2b+3ab^2-b^3$$
Разность квадратов
$$a^2-b^2=(a-b)(a+b)$$
Сумма кубов
$$a^3+b^3=(a+b)(a^2-ab+b^2)$$
Разность кубов
$$a^3-b^3=(a-b)(a^2+ab+b^2)$$

\section{Свойства степени}

$$
\begin{array}{ll}
a^0=1  & a^m\cdot a^n=a^{m+n}\\
a^m: a^n=a^{m-n} & a^{-n}=\frac{1}{a^n}\\
(a^m)^n=a^{mn} & (\frac{a}{b})^{-m}=(\frac{b}{a})^{m}\\
(a\cdot b)^m=a^m\cdot b^m & a^{\frac{1}{n}}=\root{n}\of{a}\\
(\frac{a}{b})^m =\frac{a^m}{b^m} & a^{\frac{m}{n}}=\root{n}\of{a^m}\\
\end{array}
$$

\section{Свойста квадратного (арифметического) корня}

$$
\begin{array}{ll}
\sqrt{a}\cdot \sqrt{b} = \sqrt{ab}  & \root{n}\of{a}=\root{nk}\of{a^k} \\
\frac{\sqrt{a}}{\sqrt{b}}=\sqrt{\frac{a}{b}}, b\not= 0 & \root{n}\of{a}\cdot \root{n}\of{b}=\root{n}\of{a\cdot b}\\
(\sqrt{a})^m=\sqrt{a^m} & \frac{\root{n}\of{a}}{\root{n}\of{b}}=\root{n}\of{\frac{a}{b}}, b\not=0\\
\sqrt{ab}=\sqrt{|a|}\cdot \sqrt{|b|} & (\root{n}\of{a})^m = \root{n}\of{a^m}\\
\sqrt{\frac{a}{b}}=\frac{\sqrt{|a|}}{|b|}, b\not=0 & \root{n}\of{\root{m}\of{a}} = \root{nm}\of{a}\\
\sqrt{a^m}=(\sqrt{|a|})^m
\end{array}
$$

\section{Логарифмы}
\textbf{Определение логарифма.} Логарифмом числа $b$ по основанию $a$ называется показатель степени, в которую нужно возвести $a$, чтобы получить $b.$

$$log_ab=c \Leftrightarrow a^c=b$$
 
Свойства логарифма
$$a^{log_ab}=b$$
$$log_aa=1$$
$$log_a1=0$$
$$log_aa^m=m$$

Логарифм произведения
$$log_c(ab)=log_ca+log_cb$$

Логарифм частного
$$log_c(\frac{a}{b})=log_ca-log_cb$$

Логарифм степени
$$log_ca^k=klog_ca$$

Логарифм корня
$$log_c \root{n}\of{a}=\frac{1}{n}log_ca$$

Переход к новому основанию
$$log_ab=\frac{log_cb}{log_ca}$$

Формулы, следующие из свойств логарифмов
$$log_ab=\frac{1}{log_ba}$$
$$\frac{log_nb}{log_nc}=\frac{log_mb}{log_mc}=log_cb$$
$$log_nb\cdot log_mc=log_mb\cdot log_nc$$
$$a^{log_nb}=b^{log_na}$$
\end{document}